% -*- LaTeX -*-
\documentclass{tmr}

\usepackage{mflogo}

%include polycode.fmt

\title{Combinator Parsers:  a Monad Transformer Approach}
\author{Matt Fenwick\email{mfenwick100@gmail.com}}

\begin{document}


\begin{introduction}
Parsing, or "the process of analysing a 
string of symbols according to the rules of a formal grammar" (Wikipedia),
is one of the most fundamental and interesting problems in programming.
The general goal of parsing is to build a tree representing the structure
of the parsed input; further operations such as interpretation or
code generations are then performed on the tree.

However, in addition to recognizing a stream of tokens and building a parse
tree, a practical parser has many additional duties.  If parsing fails,
the location and cause of the failure must be accurately reported so that
the user can address the problem; specially formatted comments may be need
to be capture in order to be used
to build hyperlinked code documentation, instead of simply thrown away; 
failed parses may be accompanied by partial results to save the cost
of reparsing the input before the error.

Monad transformers offer an elegant solution, both to the problem of 
implementing parser combinators themselves, as well as to the issue of
building clean and modular parsers that support multiple effects.
We'll explore what they have to offer using a simple language -- Woof --
and build a number of separate parsers for it!
\end{introduction}


\section{Parser Combinators}
Parser combinators have long been used for parsing in the 
functional programming community, represent a powerful and simple solution.
Their main advantages are ability to deal with complicated formats (see
Chomsky levels), language integration, and the declarativeness and
succinctness of the resulting parsers.
Parser combinator libraries typically include a set of basic parsers and a
set of combinators for combining little parsers into big ones.  Parsers are
typically functions whose type is something like:
\begin{verbatim}
[t] -> m (a, [t])
\end{verbatim}
where t is the token type, a is the result type, and m represents some
computational effects.
A very simple parser succeeds, consuming a single token, when the token stream
is not empty, and fails otherwise.  We can make the substitutions t -> Char,
m -> Maybe, and a -> Char and implement the function:
\begin{verbatim}
    item :: [Char] -> Maybe (Char, [Char])
    item (t:ts) = Just (t, ts)
    item []     = Nothing
    
    -- examples
    ghci$ item "abcde"
    Just ('a', "bcde")
    
    ghci$ item []
    Nothing
\end{verbatim}

We can also come up with a simple combinator that takes a parser and a function
as input, and succeeds if the parser succeeds and the function applied to the
parser's result succeeds:
\begin{verbatim}
    check :: (a -> Bool) -> ([Char] -> Maybe (a, [Char])) -> [Char] -> Maybe (a, [Char])
    check f p = p >>= \(x, ts) -> if (f x) then Just (x, ts) else Nothing
    
    -- examples
    ghci$ check (== 'a') item "abcde"
    Just ('a', "bcde")
    
    ghci$ check (== 'b') item "abcde"
    Nothing
\end{verbatim}
However, in these examples our computational effects are limited to Maybe -- we're
unable to do error reporting, logging, and so on.  Coming up with more general 
parsers and combinators will be the topic of the rest of the article.


\section{Monad transformers}
Monad transformers are a technique for combining monads to form new monads.
The motivation behind them is that while single monads are quite useful, oftentimes
the semantics of multiple monads are required simultaneously; it is then desirable
to have a modular approach for creating combined monads.

There are multiple to approaches to combining monads and to monad transformers
themselves, but we'll follow that of the standard transformers/mtl library, since
it's easy to obtain, easy to use, and quite effective in practice.

Along the way, we'll also explore some of the consequences of combining monads,
both with respect to semantics and to the definition of the Haskell language.  We'll
also bump in to some of the limitations, trade-offs and design decisions of the 
transformer/mtl library.


\section{Introducing Woof:  a Simple Lisp}
The simple language we'll use as a motivation for building parsers is 
Woof, a simple dialect of Lisp.  We'll be 
progressively adding features to a simplistic initial implementation
on our way to creating a practical, feature-rich and usable parser.

The language definition in pseudo-BNF is:

\begin{verbatim}
Woof         :=   Form(+)

Form         :=   Special  |  Application  |  Symbol  |  
                  String   |  Number

Special      :=  '{'  ( Define  |  Lambda )  '}'

Define       :=  'define'  String(?)  Symbol  Form

Lambda       :=  'lambda'  '{'  Symbol(*)  '}'  Form(+)

Application  :=  '('  Form(+)  ')'

Number       :=  \d(+)

Symbol       :=  ( \w  |  Schar)  ( \w  |  \d  |  Schar)(*)

Schar        :=  (oneof "<>!@#$%^&*_-+=|:?")

String       :=  '"'  ( Escape  |  (not  ( '"'  |  '\')))(*)  '"'

Escape       :=  '\'  ( '"'  |  '\' )

Whitespace   :=  \s+

Comment      :=  ';'  (not '\n')(*)
\end{verbatim}
With the additional rule that whitespace and comments 
may appear in any amount before any token.
Tokens are:  {, }, (, ), Symbol, String, and Number.



\section{Example 1: Recognition and Tree-Building}
The first parser we build will be responsible for determining whether input
text conforms to the language definition and simultaneously build an 
Abstract Syntax Tree representing of the structure of the recognized input.

Here's the AST definition we'll use:
\begin{verbatim}
data AST
    = ANumber Integer
    | ASymbol String
    | AString String
    | ALambda [String] [AST]
    | ADefine (Maybe String) String AST
    | AApp    AST  [AST]
  deriving (Show, Eq)
\end{verbatim}
Plus we'll need a type for our parsers.  We'll use the basic type mentioned 
above -- [t] -> m (a, [t]) -- as our starting point.  As luck would have it, that
corresponds to our first monad transformer:  StateT (Control.Monad.State), so we
can write:
\begin{verbatim}
type Parser a = StateT [Char] Maybe a
\end{verbatim}
Which means that our Parser is a State monad transformer, with the Maybe type
as its underlying Monad parameter.

Now let's get down to the business of building parsers, starting with parsers
for our most basic syntactic elements:  tokens.
These include the four braces, symbols, strings, numbers, whitespace, and comments.
To recognize braces, we need to re-implement the item and check combinators that
we did before.  Fortunately, using the StateT transformer makes it simpler
to do so:
\begin{verbatim}
item :: (MonadState [t] m, Plus m) => m t
item =
    get >>= \xs -> case xs of
                        (t:ts) -> put ts *> pure t;
                        []     -> zero;
\end{verbatim}
Although we now also need to discuss the Plus type class, why we need it instead of
Alternative, and what its instances are.


\section{Example 2: error-reporting}

\section{Example 3: whitespace/comment-logging}




\section{Transformer instance semantics}

Pass-through vs. deal-with, examples of each


\section{Limitations of the Monad Transformer Approach}

1. Exact vs. relative layer order; 
2. one instance per stack, even when more than one is applicable --
how to specify which one?
3. an inappropriate instance higher-up in the stack can mask a 
desired instance lower down in the stack -- that's why we had to
create a replacement for the Alternative type class and provide
some new instances for it -- to be able to use the right instances



\section{Conclusion}

Parser combinators and monad transformers FTW!


\bibliography{Author}

\end{document}

% -*- LaTeX -*-
\documentclass{tmr}

\usepackage{mflogo}

%include polycode.fmt

\title{Combinator Parsers:  a Monad Transformer Approach}
\author{Matt Fenwick\email{mfenwick100@gmail.com}}

\begin{document}


\begin{introduction}
Parsing, or "the process of analysing a 
string of symbols according to the rules of a formal grammar" (Wikipedia),
is one of the most fundamental and interesting problems in programming.
The general goal of parsing is to build a tree representing the structure
of the parsed input; further operations such as interpretation or
code generations are then performed on the tree.

However, in addition to recognizing a stream of tokens and building a parse
tree, a practical parser has many additional duties.  If parsing fails,
the location and cause of the failure must be accurately reported so that
the user can address the problem; specially formatted comments may be need
to be capture in order to be used
to build hyperlinked code documentation, instead of simply thrown away; 
failed parses may be accompanied by partial results to save the cost
of reparsing the input before the error.

Monad transformers offer an elegant solution, both to the problem of 
implementing parser combinators themselves, as well as to the issue of
building clean and modular parsers that support multiple effects.
We'll explore what they have to offer using a simple language -- Woof --
and build a number of separate parsers for it!
\end{introduction}


\section{Parser Combinators}
Parser combinators have long been used for parsing in the 
functional programming community, represent a powerful and simple solution.
Their main advantages are ability to deal with complicated formats (see
Chomsky levels), language integration, and the declarativeness and
succinctness of the resulting parsers.
Parser combinator libraries typically include a set of basic parsers and a
set of combinators for combining little parsers into big ones.  Parsers are
typically functions whose type is something like:
\begin{verbatim}
[t] -> m ([t], a)
\end{verbatim}
where t is the token type, a is the result type, and m represents some
computational effects.
A very simple parser succeeds, consuming a single token, when the token stream
is not empty, and fails otherwise.  We can make the substitutions t -> Char,
m -> Maybe, and a -> Char and implement the function:
\begin{verbatim}
    item :: [Char] -> Maybe ([Char], Char)
    item (t:ts) = Just (ts, t)
    item []     = Nothing
    
    -- examples
    ghci$ item "abcde"
    Just ("bcde", 'a')
    
    ghci$ item []
    Nothing
\end{verbatim}

We can also come up with a simple combinator that takes a parser and a function
as input, and succeeds if the parser succeeds and the function applied to the
parser's result succeeds:
\begin{verbatim}
    check :: (a -> Bool) -> ([Char] -> Maybe ([Char], a)) -> [Char] -> Maybe ([Char], a)
    check f p = p >>= \(ts, x) -> if (f x) then Just (ts, x) else Nothing
    
    -- examples
    ghci$ check (== 'a') item "abcde"
    Just ("bcde", 'a')
    
    ghci$ check (== 'b') item "abcde"
    Nothing
\end{verbatim}
However, in these examples our computational effects are limited to Maybe -- we're
unable to do error reporting, logging, and so on.  Coming up with more general 
parsers and combinators will be the topic of the rest of the article.


\section{Monad transformers}

\section{Introducing Woof:  a Simple Lisp}

The following examples will all use Woof, a simple dialect of Lisp,
progressively adding features to a simplistic initial implementation
to create a usable parser.

The language definition in pseudo-BNF is:

\begin{verbatim}
Woof         :=   Form(+)

Form         :=   Special  |  Application  |  Symbol  |  
                  String   |  Number

Special      :=  '{'  ( Define  |  Lambda )  '}'

Define       :=  'define'  String(?)  Symbol  Form

Lambda       :=  'lambda'  '{'  Symbol(*)  '}'  Form(+)

Application  :=  '('  Form(+)  ')'

Number       :=  \d(+)

Symbol       :=  ( \w  |  Schar)  ( \w  |  \d  |  Schar)(*)

Schar        :=  (oneof "<>!@#$%^&*_-+=|:?")

String       :=  '"'  ( Escape  |  (not  ( '"'  |  '\')))(*)  '"'

Escape       :=  '\'  ( '"'  |  '\' )

Whitespace   :=  \s+

Comment      :=  ';'  (not '\n')(*)
\end{verbatim}
With the additional rule that whitespace and comments 
may appear in any amount before any token.
Tokens are:  {, }, (, ), Symbol, String, and Number.



\section{Example 1: recognizing}

\subsection{An Elementary Parser}
The basic types of parsers will be something like
\begin{verbatim}[t] -> m ([t], a)\end{verbatim}, or, a function that consumes
part of a list of tokens of type t with computational effects m and
result a.
A very basic and fundamental parser is the one that succeeds whenever
the list of tokens is not empty, consuming one token and presenting it
as the result, and fails when the list is empty.  We'll call this item,
and we can easily implement it in terms of the MonadState type class:
\begin{verbatim}
item :: (MonadState [t] m, Plus m) => m t
item =
    get >>= \xs -> case xs of
                        (t:ts) -> put ts *> pure t;
                        []     -> zero;
\end{verbatim}

\subsection{Token Parsers}
Let's start with parsers for recognizing our most basic syntactic elements:  tokens.
These include the seven listed above, as well as whitespace and comments.


\section{Example 2: error-reporting}

\section{Example 3: whitespace/comment-logging}




\section{Transformer instance semantics}

Pass-through vs. deal-with, examples of each


\section{Limitations of the Monad Transformer Approach}

1. Exact vs. relative layer order; 
2. one instance per stack, even when more than one is applicable --
how to specify which one?
3. an inappropriate instance higher-up in the stack can mask a 
desired instance lower down in the stack -- that's why we had to
create a replacement for the Alternative type class and provide
some new instances for it -- to be able to use the right instances



\section{Conclusion}

Parser combinators and monad transformers FTW!


\bibliography{Author}

\end{document}

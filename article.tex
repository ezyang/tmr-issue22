% -*- LaTeX -*-
\documentclass{tmr}

\usepackage{mflogo}

%include polycode.fmt

\title{Error Reporting Parsers:  a Monad Transformer Approach}
\author{Matt Fenwick\email{mfenwick100@@gmail.com}}

\begin{document}


\begin{introduction}
Parsing is "the process of analysing a string of symbols according to the rules
of a formal grammar" \cite{wikipedia_parsing}, with the general goal of building 
a tree representing the structure of the parsed input, upon which further 
operations such as interpretation or code generation are then performed.

In addition to recognizing a stream of tokens and building a parse tree, a
real-world parser must recognize faulty input, and accurately report the
location and cause of the problem.  But how -- what generates a parse error,
what information is included in the error, how do error-reporting parsers
compose?

Monad transformers provide a clean and simple answer to these questions.
We'll use them to build a parser combinator library, then we'll use the library
to build a simple parser, and finally we'll add in error reporting.
\end{introduction}




\section{Parser Combinators}
Parser combinators are an excellent approach for building expressive parsers 
that are easy to maintain.  In addition, they benefit from host-language
integration, allowing them to snarf features such as type systems and test 
frameworks.

There are many excellent papers on parser combinators, such
as this \cite{wadler} classic Wadler paper on non-deterministic parser combinators, 
Hutton's paper which covers factoring parsers into smaller pieces \cite{hutton}, 
and Leijen's Parsec paper \cite{leijen}.  
The inspiration and and much of the names used in this article are drawn 
from those references.

Let's make a quick pass through a minimal parser combinator library so that we
can see them in action before we move on to the hard stuff.  There are three
main pieces of the code:
\begin{itemize}
 \item a Parser datatype
 \item parser-specific combinators
 \item type class instances
\end{itemize}

\subsection{Parser datatype}
\begin{verbatim}
newtype Parser t a = 
    Parser {getParser :: [t] -> Maybe ([t], a)}
\end{verbatim}
This type definition says that a Parser is a function
of a list of tokens, producing a new list of tokens and a result
value, and possibly failing.

\subsection{Type class instances}
For running parsers in sequence, we'll use the combinators from the Monad and
Applicative type classes.  There's a little bit of book-keeping involved in the
Monad instance -- it has to make sure that the correct token list is used by
each parser -- but once we have a Monad instance, the Applicative and Functor
instances can be handled by imports \verb+ap+ and \verb+liftM+ from Control.Monad.
\begin{verbatim}
import Control.Applicative  (Applicative(..), Alternative(..))
import Control.Monad        (liftM, ap)

instance Monad (Parser t) where
  return x = Parser (\ts -> Just (ts, x))
  Parser p >>= f = Parser (\ts ->
                             p ts >>= \(ts', x) ->
                             getParser (f x) ts')

instance Functor (Parser t) where
  fmap = liftM
  
instance Applicative (Parser t) where
  pure = return
  (<*>) = ap 
\end{verbatim}

We use the Alternative type class for expressing choice; the Parser instance is  
is defined in terms of the Alternative instance of the
underlying result type, \verb+Maybe+.
\begin{verbatim}
instance Alternative (Parser t) where
  empty = Parser (const empty)
  Parser l <|> Parser r = Parser (\ts -> l ts <|> r ts)
\end{verbatim}

\subsection{Parser-specific combinators}
\verb+item+ is a very simple parser; it 
consumes a single token if available, and fails otherwise:
\begin{verbatim}
item :: Parser t t
item = Parser f
  where 
    f [] = Nothing
    f (x:xs) = Just (xs, x)
\end{verbatim}

Analagous to \verb+mfilter+ is the \verb+check+ combinator, which is used to
validate a parse result:
\begin{verbatim}
check :: (Monad f, Alternative f) => (a -> Bool) -> f a -> f a
check p m =
    m >>= \x -> if p x then return x else empty
\end{verbatim}

Using \verb+check+, we can build a very convenient combinator to parse specific
matching tokens: 
\begin{verbatim}
literal :: Eq t => t -> Parser t t
literal x = check (== x) item
\end{verbatim}

\subsection{Examples}
Now for some examples:  first, let's see \verb+item+ in action.  It fails on empty lists,
and succeeds consuming one token otherwise:
\begin{verbatim}
*Main> getParser item ""
Nothing

*Main> getParser item "abcde"
Just ("bcde",'a')
\end{verbatim}

We can run parsers in sequence using the Applicative combinators.  First one parser
is run, then the second, and the token position is threaded between the two.
Note that both parsers must succeed in order for the combined parser to succeed:
\begin{verbatim}
*Main> getParser (item *> item) "abcde"
Just ("cde",'b')

*Main> getParser (fmap (,) item <*> item) "abcde"
Just ("cde",('a','b'))

*Main> getParser (item *> item) "a"
Nothing
\end{verbatim}

Parsers built out of \verb+literal+ only succeed when the next token matches 
the given one:
\begin{verbatim}
*Main> getParser (literal 'a') "abcde"
Just ("bcde",'a')

*Main> getParser (literal 'a') "bcde"
Nothing
\end{verbatim}

Using the Alternative combinators, we can create parsers that succeed if any of
their sub-parsers succeed.  This is what allows parsers to backtrack, trying
various alternatives if one fails:
\begin{verbatim}
*Main> getParser (literal 'a' <|> literal 'b') "abcde"
Just ("bcde",'a')

*Main> getParser (literal 'a' <|> literal 'b') "bacde"
Just ("acde",'b')

*Main> getParser (literal 'a' <|> literal 'b') "cdeab"
Nothing
\end{verbatim}
In the rest of this article, we'll be reworking these basic parser combinators,
by improving the implementations and extending the result types,
all while taking advantage of monad transformers to keep it modular.




\section{Monad transformers}
Monad transformers \cite{liang} allow monads to be modularly combined to form 
new, more complicated monads; the combined monads support all the operations of
their constituents.  Doing just that -- combining multiple monads into a stack --
we can build semantically rich yet modular parsers.
We'll try to use the standard transformers/mtl \cite{mtl} libraries as much as possible, 
since they're easy to obtain, easy to use, and quite effective in practice, but
occasionally we'll have to make a slight modification or extensions.

First, note that the parser type we used earlier -- \verb+[t] -> Maybe ([t], a)+
-- can be built out of the StateT monad transformer applied to Maybe.  We can 
now define the new parser type as:
\begin{verbatim}
type Parser t a = StateT [t] Maybe a
\end{verbatim}

We can build the \verb+item+ parser using the combinators from MonadState --
token lists are the 'state' that \verb+item+ must inspect and modify:
\begin{verbatim}
{-# LANGUAGE   FlexibleContexts  #-}

import Control.Monad.State       (MonadState (..), StateT(..))

item :: (MonadState [t] m, Alternative m) => m t
item =
    get >>= \xs -> case xs of
                        (t:ts) -> put ts *> pure t;
                        []     -> empty;
\end{verbatim}
Note that we needed to use the FlexibleContexts language extension for the 
MonadState constraints in the type signatures.  This formulation of monad
transformers relies heavily on extensions to the type system, more of which 
we'll have to use for some of the later code in this article.




\section{Introducing Woof:  a Simple Lisp}
The simple language we'll use as a motivation for building parsers is Woof, a 
simple dialect of Lisp.  We'll be progressively adding features to a simple 
initial implementation on our way to creating an error-reporting parser.

The language definition in pseudo-BNF \cite{bnf} is:

\begin{verbatim}
Woof         :=   Form(+)

Form         :=   Symbol  |  Special  |  Application

Symbol       :=   [a-zA-Z](+)

Special      :=   '{'  ( Define  |  Lambda )  '}'

Define       :=   'define'  Symbol  Form

Lambda       :=   'lambda'  '{'  Symbol(*)  '}'  Form(+)

Application  :=   '('  Form(+)  ')'

Whitespace   :=   \s+

Comment      :=   ';'  (not '\n')(*)
\end{verbatim}

With the additional rule that whitespace and comments may appear in any 
amount before any token.  Tokens are:
\begin{itemize}
  \item \verb+{+
  \item \verb+}+
  \item \verb+(+
  \item \verb+)+
  \item \verb+Symbol+
\end{itemize}




\section{Example 1: Recognition and Tree-Building}

\subsection{Preliminaries}
Our first Woof parser will be responsible for: 1) determining whether input
text conforms to the language definition, and 2) building an 
Abstract Syntax Tree representing the structure of the parsed input.
Here's the AST definition we'll use:
\begin{verbatim}
data AST
    = ASymbol String
    | ALambda [String] [AST]
    | ADefine String AST
    | AApp    AST  [AST]
  deriving (Show, Eq)
\end{verbatim}

We'll also want a function for running our parsers, using the type given in the
prevous section:
\begin{verbatim}
runParser :: Parser [t] a -> [t] -> Maybe (a, [t])
runParser = runStateT
\end{verbatim}

\subsection{Token parsers}
Now let's get down to the business of building parsers, starting with parsers
for our most basic syntactic elements:  tokens.
These include the four braces, symbols, strings, numbers, whitespace, and comments.
To recognize braces, we need to use the \verb+item+ and \verb+check+ 
combinators to build a new combinator, \verb+satisfy+, that checks whether 
the next character meets a condition:
\begin{verbatim}
satisfy :: (MonadState [t] m, Alternative m) => (t -> Bool) -> m t
satisfy = flip check item
\end{verbatim}

And here's the slightly modified \verb+literal+ combinator:
\begin{verbatim}
literal :: (Eq t, MonadState [t] m, Alternative m) => t -> m t
literal c = satisfy ((==) c)
\end{verbatim}

Now we're ready to build parsers for the four bracket tokens.  
First, a simple bracket parser:
\begin{verbatim}
opencurly = literal '{'
\end{verbatim}

And some examples to make sure it's behaving correctly:
\begin{verbatim}
*Main> map (runParser  opencurly) ["{abc", "}abc", "(abc", "abc"]
[Just ('{', "abc"), Nothing, Nothing, Nothing]
\end{verbatim}

Good, it correctly accepts open curly brackets and rejects everything else.
But what about the spec that said that comments and whitespace can occur 
before any token -- will our parser be able to handle that?
\begin{verbatim}
*Main> map (runParser  opencurly) ["}abc", "(abc", "abc"]
Nothing
\end{verbatim}

Of course not -- we haven't told it how to skip comments and whitespace yet!
Let's do that now, first by defining the whitespace and comment patterns, then
by creating a \verb+munch+ combinator that takes in a parser, 
throws away all leading junk, and finally runs the parser:
\begin{verbatim}
whitespace = many1 $ satisfy (flip elem " \n\t\r\f")
comment = pure (:) <*> literal ';' <*> many0 (not1 $ literal '\n')

munch p = many0 (whitespace <+> comment) *> p
\end{verbatim}

Here are these three parsers in action:
\begin{verbatim}
*Main> runParser whitespace "\n \t   \tabc"
Just ("\n \t   \t","abc")

*Main> runParser whitespace "abc"
Nothing

*Main> runParser comment ";123  \t\nabc"
Just (";123  \t","\nabc")

*Main> runParser (munch $ literal 'a') "abc"
Just ('a',"bc")

*Main> runParser (munch $ literal 'a') ";comment!\n   abc"
Just ('a',"bc")

*Main> runParser (munch $ literal 'a') ";comment!\n   bca"
Nothing

*Main> runParser (munch $ literal 'a') "bca"
Nothing
\end{verbatim}

The comment parser uses a new combinator, \verb+not1+, built out of a 
type class, Switch, modeling computations which can be switched from
successful to failing and vice versa:
\begin{verbatim}
class Switch f where
  switch :: f a -> f ()

instance Switch Maybe where
  switch (Just _) = Nothing
  switch Nothing  = Just ()

instance (Functor m, Switch m) => Switch (StateT s m) where
  switch (StateT f) = StateT g
    where 
      g s = fmap (const ((), s)) . switch $ f s

not1 :: (MonadState [t] m, Alternative m, Switch m) => m a -> m t
not1 p = switch p *> item
\end{verbatim}

With \verb+munch+ in hand, we can now fix \verb+opencurly+ and try it out:
\begin{verbatim}
opencurly = munch $ literal '{'

*Main> runParser opencurly ";abc\n{def"
Just ('{', "def")

*Main> runParser opencurly "{def"
Just ('{', "def")

*Main> runParser opencurly "(def"
Nothing

*Main> runParser opencurly "}def"
Nothing
\end{verbatim}

Excellent!  Now the \verb+opencurly+ parser correctly throws away whitespace 
and comments and succeeds, consuming one character, when the first non-junk 
character is an open curly brace, failing otherwise.
It's straightforward to define parsers for the other three brace tokens:
\begin{verbatim}
closecurly = munch $ literal '}'
openparen  = munch $ literal '('
closeparen = munch $ literal ')'
\end{verbatim}

The parser for our last token, Symbol, is slightly more complicated because it allows
multiple characters and multiple lengths.  We'll use the satisfy combinator along with 
a function that checks whether a character is one of the legal symbol characters:
\begin{verbatim}
symbol = munch $ many1 char
  where char = satisfy (flip elem (['a' .. 'z'] ++ ['A' .. 'Z']))

*Main> runParser symbol "abc123"
Just ("abc","123")

*Main> runParser symbol ";sdfasdfsa\n  abc123"
Just ("abc","123")

*Main> runParser symbol ";sdfasdfsa\n  123abc"
Nothing
\end{verbatim}

Note the use of \verb+many1+ to introduce repetition,
and we're again using \verb+munch+ to throw away leading junk.

\subsection{Syntactic structures}
Whereas our token parsers dealt with the syntactic primitives of the Woof grammar, 
the remaining parsers will have to implement grammar rules that combine the tokens
into syntactic structures.
The first combining form is function application.  The rule says that it's delimited
by matching parentheses, in between which must appear one form as the operator, followed
by any number of forms as the arguments to which the operator is applied.  We can
say that like so:
\begin{verbatim}
application =
    openparen    *>
    pure AApp   <*>
    form        <*>
    many0 form  <*
    closeparen
\end{verbatim}

Note that all the parser sequencing is done using Applicative
combinators -- we don't need to use the monadic ones.  Also, we used \verb+pure+ 
to inject a value -- the function \verb+AApp+ -- into the parser.  \verb+pure+ 
has no effect on the tokens and always succeeds:
\begin{verbatim}
*Main> runParser (pure 3) ""
Just (3,"")

*Main> runParser (pure 3) "abc"
Just (3,"abc")
\end{verbatim}

Next, we move on to \verb+define+.
A straightforward translation from the grammar produces:
\begin{verbatim}
define =
    opencurly                    *>
    check (== "define") symbol   *>
    pure ADefine                <*>
    symbol                      <*>
    form                        <*
    closecurly
\end{verbatim}

On to Lambda.  Not only does Lambda have additional syntax with its parameter 
list, but we also need to ensure that the parameter names are unique.  We can 
do that using the \verb+check+ combinator again:
\begin{verbatim}
lambda = 
    opencurly                       *>
    check (== "lambda") symbol      *>
    opencurly                       *>
    pure ALambda                   <*>
    check distinct (many0 symbol)  <*>
    (closecurly                     *>
     many1 form                    <*
     closecurly)
  where
    distinct names = length names == length (nub names)
\end{verbatim}
If the symbols are distinct, the parameter list subparser will succeed,
whereas if there's a repeated symbol, it will fail.

\subsection{Finishing touches}
All we have still to do is to define the \verb+special+, \verb+form+ (which 
we've already used to build the previous parsers), and \verb+woof+ parsers.  
The special parser is simple -- it parses either define or lambda:
\begin{verbatim}
special = define <+> lambda
\end{verbatim}

A form is either a symbol, application, or a special form.  We could just write:
\begin{verbatim}
form = symbol <+> application <+> special
\end{verbatim}

except that the types don't match -- \verb+symbol+'s type parameter is a String, 
whereas the other two have ASTs.  We fix that by mapping the \verb+ASymbol+ 
function over \verb+symbol+:
\begin{verbatim}
form = fmap ASymbol symbol <+> application <+> special
\end{verbatim}

The final parser, \verb+woof+, needs to not only parse all the forms, but also 
make sure that the entire input is consumed.  The end of input check is done 
with the \verb+end+ parser:
\begin{verbatim}
end = switch item
\end{verbatim}

We also need to munch any trailing comments and whitespace -- otherwise \verb+end+
would fail -- so the final woof parser is:
\begin{verbatim}
woof = many1 form <* munch end
\end{verbatim}

\subsection{Examples}
Let's look at a few examples of our parser in action:
\begin{verbatim}
*Main> runParser woof "; \n   {lambda {x y z} a b (c b a)}end"
Just ([ALambda ["x","y","z"] 
               [ASymbol "a",
                ASymbol "b",
                AApp (ASymbol "c") 
                     [ASymbol "b",
                      ASymbol "a"]],
       ASymbol "end"],
      "")

*Main> runParser woof "a b c {define q (f x)}"
Just ([ASymbol "a",
       ASymbol "b",
       ASymbol "c",
       ADefine "q" (AApp (ASymbol "f") [ASymbol "x"])],"")

*Main> runParser woof "a b c {define q (f x)},"
Nothing
\end{verbatim}
Note that the first two parses consume the entire input, while the last one 
fails because it doesn't recognize the trailing comma.




\section{Example 2: error-reporting}
While our first parser worked great on valid input, it wasn't helpful
when the input was malformed.  When the parser finds bad input, we'd like it to 
produce an informative error, indicating what and where the problem was.  
This allows a user to quickly find and correct problems.

Let's start with an informal spec of different cases of malformed input, along 
with the error messages that should be reported: 
\begin{itemize}
  \item application: missing operator:  \begin{verbatim}()\end{verbatim}
  \item application: missing close parenthesis:  \begin{verbatim}(a b\end{verbatim}
  \item define: missing symbol:  \begin{verbatim}{define (a b c)}\end{verbatim}
  \item define: missing form:  \begin{verbatim}{define a}\end{verbatim}
  \item define: missing close curly:  \begin{verbatim}{define a b\end{verbatim}
  \item lambda: missing parameter list:  \begin{verbatim}{lambda (a b c)}\end{verbatim}
  \item lambda: duplicate parameter names:  \begin{verbatim}{lambda {a b a} (c d)}\end{verbatim}
  \item lambda: missing parameter list close curly:  \begin{verbatim}{lambda {a b (c d)}\end{verbatim}
  \item lambda: missing body form:  \begin{verbatim}{lambda {a b}}\end{verbatim}
  \item lambda: missing close curly:  \begin{verbatim}{lambda {a b} (c d)\end{verbatim}
  \item special form: unable to parse:  \begin{verbatim}{defin x y}\end{verbatim}
  \item woof: unparsed input:  \begin{verbatim}a,b\end{verbatim}
\end{itemize}
To allow error reporting in our parsers, we first need to extend our monad stack, 
adding in a layer for errors.  We'll use a second monad transformer type class 
-- MonadError, replace the Maybe layer with its corresponding transformer datatype 
MaybeT, and add in an Either layer on the bottom of the stack.  Our parser stack 
and \verb+runParser+ function are now:
\begin{verbatim}
type Parse e t a = StateT [t] (MaybeT (Either e)) a

runParser :: Parse e t a -> [t] -> Either e (Maybe (a, [t]))
runParser p xs = runMaybeT (runStateT p xs)
\end{verbatim}

While our previous parsers had two different results -- Nothing and Just ... --
our new parsers have three possible results:  success, failure, and error.  Note
that successful and failing parses are wrapped in an additional \verb+Right+
constructor:
\begin{verbatim}
*Main> runParser item "abc"
Right (Just ('a',"bc"))

*Main> runParser item ""
Right Nothing

*Main> runParser (throwError "oops!") ""
Left "oops!"
\end{verbatim}

Although the \verb+<|>+ combinator can recover from failures, it can't recover from
errors.  Compare:
\begin{verbatim}
*Main> runParser (satisfy (== 'a') <|> satisfy (== 'b')) "babc"
Right (Just ('b',"abc"))

*Main> runParser (throwError "no!" <|> satisfy (== 'b')) "babc"
Left "no!"
\end{verbatim}
We'll take advantage of this backtracking restriction to produce accurate error
messages.

\subsection{Type Class Preliminaries}
Unfortunately, the standard MonadError instance for the \verb+Either+ datatype does
not fit our needs -- it requires an \verb+Error+ constraint that we do not want.
We'll implement our own MonadError type class, based on the
standard one, and instances of it.  We'll need an instance for each member of
our stack -- Either, StateT, and MaybeT: 
\begin{verbatim}
class Monad m => MonadError e m | m -> e where
  throwError :: e -> m a
  catchError :: m a -> (e -> m a) -> m a

instance MonadError e (Either e) where
  throwError               =  Left
  catchError  (Right x) _  =  Right x
  catchError  (Left e)  f  =  f e
  
instance MonadError e m => MonadError e (StateT s m) where
  throwError      =  lift . throwError
  catchError m f  =  StateT g
    where
      g s = catchError (runStateT m s) 
                       (\e -> runStateT (f e) s)

instance MonadError e m => MonadError e (MaybeT m) where
  throwError      =  lift . throwError
  catchError m f  =  MaybeT $ catchError (runMaybeT m) 
                                         (runMaybeT . f)
\end{verbatim}
Note how the StateT and MaybeT instances don't really do anything with the
errors; they just pass the error on through to the next level.  That's why they
both require that the next lower level supports MonadError.  Also, this code 
requires some additional extensions:
\begin{itemize}
  \item FlexibleInstances
  \item FunctionalDependencies
  \item UndecidableInstances 
\end{itemize}

We'll also need an instance of Switch for MaybeT:
\begin{verbatim}
instance Functor m => Switch (MaybeT m) where
  switch (MaybeT m) = MaybeT (fmap switch m)
\end{verbatim}

\subsection{Reporting position}
To report error positionr, we'll pre-process the input
string to calculate the line and column number of each character.  Then instead 
of our parsers operating on a list of characters, they'll operate on a list of 
Tokens, where a Token is a character tupled up with a line number and column 
number, which we'll express with a simple typedef and some accessor functions:
\begin{verbatim}
type Token = (Char, Int, Int)
chr  (a, _, _)  =  a
line (_, b, _)  =  b
col  (_, _, c)  =  c
\end{verbatim}
The function responsible for calculating the position will simply transform a 
list of characters into a list of tokens:
\begin{verbatim}
countLineCol :: [Char] -> [Token]
countLineCol = reverse . snd . foldl f ((1, 1), [])
  where
    f ((line, col), ts) '\n' = ((line + 1, 1), ('\n', line, col):ts)
    f ((line, col), ts)  c   = ((line, col + 1), (c, line, col):ts)
\end{verbatim}

\subsection{Token parsers}
By adding the position to the input, we've created a new problem 
for ourselves: our original token parsers operated on 
character lists, but we now need them to work on token lists.  
The cause of the problem is the \verb+literal+ combinator, so we
replace it with a similar one for token lists:
\begin{verbatim}
character :: (MonadState [Token] m, 
              Alternative m) => Char -> m Token
character c = satisfy ((==) c . chr)
\end{verbatim}

And now our token parsers are:
\begin{verbatim}
whitespace = many1 $ satisfy (flip elem " \n\t\r\f" . chr)
comment = pure (:) <*> character ';' <*> many0 chars
  where chars = not1 $ character '\n'

munch p = many0 (whitespace <+> comment) *> p

ocurly = munch $ character '{'
ccurly = munch $ character '}'
oparen = munch $ character '('
cparen = munch $ character ')'
symbol = munch $ many1 char
  where char = fmap chr $ satisfy (f . chr)
        f = flip elem (['a' .. 'z'] ++ ['A' .. 'Z'])
\end{verbatim}

The only differences were the replacement of \verb+literal+ with \verb+character+ 
and a couple uses of the \verb+chr+ accessor.  The \verb+munch+ combinator didn't 
have to change at all (although its inferred type did in fact change).

\subsection{Error messages}
Now we modify the remaining parsers to report accurate and useful errors.
Here are the error messages that we'll use to report each error:
\begin{verbatim}
eAppOper    =  "application: missing operator"
eAppClose   =  "application: missing close parenthesis"
eDefSym     =  "define: missing symbol"
eDefForm    =  "define: missing form"
eDefClose   =  "define: missing close curly"
eLamParam   =  "lambda: missing parameter list"
eLamDupe    =  "lambda: duplicate parameter names"
eLamPClose  =  "lambda: missing parameter list close curly"
eLamBody    =  "lambda: missing body form"
eLamClose   =  "lambda: missing close curly"
eSpecial    =  "special form: unable to parse"
eWoof       =  "woof: unparsed input"
\end{verbatim}

The MonadError type class provides two useful combinators, \verb+throwError+ for 
generating errors, and \verb+catchError+ for dealing with them.  We can ignore 
the second because we don't need to recover from errors, just report them, 
but when should an error be generated and what 
data should it include?  \verb+throwError+ is too basic for our needs, so we'll 
introduce the \verb+commit+ combinator:
\begin{verbatim}
commit :: (MonadError e m, Alternative m) => e -> m a -> m a
commit err p = p <|> throwError err
\end{verbatim}

This combinator takes two arguments: 1. an error value, and 2. a monadic 
computation.  It first tries to run the computation, and if the computation 
fails, generates an error whose contents are the first argument.  The 
definition of 'failure' is provided by the Alternative instance of \verb+m+.

What data do we want to include in our errors?  A String describing the error 
and a token reporting the position will describe the error decently:
\begin{verbatim}
type Error = (String, Token)
\end{verbatim}

Now let's start with \verb+application+ as our first modified, error-reporting 
parser:
\begin{verbatim}
application =
    openparen                             >>= \open ->
    commit (eAppOper, open) form          >>= \op ->
    many0 form                            >>= \args ->
    commit (eAppClose, open) closeparen   >>
    return (AApp op args)
\end{verbatim}
This parser works by running the open-parenthesis parser, capturing its value 
in \verb+open+, and continuing with the parsing; if either \verb+form+ or 
close-parenthesis fails, an appropriate error is generated with a message and 
the location of the open parenthesis.  However, if the open 
parenthesis is not successfully parsed, it's a failure but not an error.  
Examples:
\begin{verbatim}
*Main> runParser application (countLineCol "oops")
Right Nothing

*Main> runParser application (countLineCol "(a b c)")
Right (Just (AApp (ASymbol "a") [ASymbol "b",ASymbol "c"],[]))

*Main> runParser application (countLineCol "\n()")
Left ("application: missing operator",('(',2,1))

*Main> runParser application (countLineCol "(a b")
Left ("application: missing close parenthesis",('(',1,1))
\end{verbatim}

Although the previous version of the \verb+application+ parser was implemented 
with just applicative combinators, this version can't be:  \verb+open+ 
is used to build the following parsers, making this monadic.

The \verb+define+ and \verb+lambda+ parsers are upgraded similarly:
\begin{verbatim}
define =
    opencurly                            >>= \open ->
    check (== "define") symbol            *>
    pure ADefine                         <*>
    commit (eDefSym, open) symbol        <*>
    commit (eDefForm, open) form         <*
    commit (eDefClose, open) closecurly  

lambda =
    opencurly                                >>= \open ->
    check (== "lambda") symbol               >>
    commit (eLamParam, open) opencurly       >>= \p_open ->
    many0 symbol                             >>= \params ->
    (if distinct params 
        then return ()
        else throwError (eLamDupe, p_open))  >>
    commit (eLamPClose, p_open) closecurly   >>
    commit (eLamBody, open) (many1 form)     >>= \bodies ->
    commit (eLamClose, open) closecurly      >>
    return (ALambda params bodies)
  where
    distinct names = length names == length (nub names)
\end{verbatim}

Note that \verb+commit+ isn't used until we can be sure that we're in the right 
rule:  if, in the \verb+define+ parser, we committed to parsing the "define" 
keyword, we wouldn't be able to backtrack and try the \verb+lambda+ parser if 
that failed.  Thus, if errors are treated as unrecoverable, it's important not to 
place a \verb+commit+ where backtracking might be necessary to parse a valid 
alternative.  However, in our example, once 
we see an open curly brace and a define, we're sure that we're in the \verb+define+
rule, and allowing useless backtracking would only destroy our ability to report 
errors cleanly and accurately.

\subsection{Error messages:  finishing touches}
To complete the error-reporting parser, we need to change a couple more details. 
First, we want an error reported when an open curly brace is found, but no 
special form can be parsed.  We extend the \verb+special+ 
parser with a third alternative, which throws an error if it can parse
an open curly brace:
\begin{verbatim}
special = define   <+> 
          lambda   <+> 
          (ocurly >>= \o -> throwError (esName, o))
\end{verbatim}

The \verb+form+ parser doesn't have to be changed:
\begin{verbatim}
form = fmap ASymbol symbol <+> application <+> special
\end{verbatim}

When we can no longer parse any forms, instead of failing, we'd like to report 
an error if there's any input left, and otherwise succeed.  Our ending parser
must first munch any leading whitespace or comments, as usual, and then check 
whether there are any tokens left.  If there are, it should report an error at 
the location of the first remaining token:
\begin{verbatim}
endCheck = 
    many0 (whitespace <+> comment) *>
    get >>= \xs -> case xs of
                        (t:_) -> throwError (ewUnp, t)
                        []    -> pure ()
\end{verbatim}
The \verb+woof+ parser uses \verb+endCheck+ after consuming all possible forms:
\begin{verbatim}
woof = many0 form <* endCheck
\end{verbatim}

\subsection{Error messages:  examples}
While correct input is still parsed successfully:
\begin{verbatim}
*Main> let p = runParser woof . countLineCol

*Main> p "{define fgh {lambda {f x} (f x)}}"
Right (Just ([ADefine "fgh" (ALambda ["f","x"] 
                                     [AApp (ASymbol "f") 
                                           [ASymbol "x"]])],
       []))
\end{verbatim}

Incorrect input is reported as an error.  For each of the items in our error spec, 
we can show that our parser correctly reports an error: 
\begin{verbatim}
*Main> p "()"
Left ("application: missing operator",('(',1,1))

*Main> p "(a b"
Left ("application: missing close parenthesis",('(',1,1))

*Main> p "{define (a b c)}"
Left ("define: missing symbol",('{',1,1))

*Main> p "{define a}"
Left ("define: missing form",('{',1,1))

*Main> p "{define a b"
Left ("define: missing close curly",('{',1,1))

*Main> p "{lambda (a b c)}"
Left ("lambda: missing parameter list",('{',1,1))

*Main> p "{lambda {a b a} (a b c)}"
Left ("lambda: duplicate parameter names",('{',1,9))

*Main> p "{lambda {a b (c d)}"
Left ("lambda: missing parameter list close curly",('{',1,9))

*Main> p "{lambda {a b}}"
Left ("lambda: missing body form",('{',1,1))

*Main> p "{lambda {a b} (c d)"
Left ("lambda: missing close curly",('{',1,1))

*Main> p "{defin x y}"
Left ("special form: unable to parse",('{',1,1))

*Main> p "a,b"
Left ("woof: unparsed input",(',',1,2))
\end{verbatim}




\section{Stack Order}
Not only are the contents of a transformer stack important, but the order of the
layers as well \cite{stack}.  Let's revisit our first woof parser -- what if our 
stack had been Maybe/State instead of State/Maybe?
\begin{verbatim}
State/Maybe:   s -> Maybe (a, s)

Maybe/State:   s -> (Maybe a, s)
\end{verbatim}

given:
\begin{verbatim}
import Control.Monad.Trans.Maybe (MaybeT(..))
import Data.Functor.Identity     (Identity(..))

type Parser' t a = MaybeT (StateT [t] Identity) a

runParser' :: Parser' t a -> [t] -> (Maybe a, [t])
runParser' p xs = runIdentity $ runStateT (runMaybeT p) xs
\end{verbatim}

We can compare it to the earlier parser:
\begin{verbatim}
*Main> runParser' (literal 'a') "aqrs"
(Just 'a',"qrs")

*Main> runParser' (literal 'a') "qrs"
(Nothing,"rs")
\end{verbatim}
Whereas:
\begin{verbatim}
*Main> runParser (literal 'a') "qrs"
Nothing

*Main> runParser (literal 'a') "aqrs"
Just ('a',"qrs")
\end{verbatim}

The difference is that Maybe/State always returns a modified state, regardless
of the success of the computation, whereas State/Maybe only returns a modified
state if the computation is successful.  This means that backtracking doesn't
work right with Maybe/State -- tokens are consumed even when the parsers
don't match, as the example shows.

On the other hand, the order of Error and Maybe relative to each other isn't 
important.  But don't just take my word for it -- try it for yourself!

Another important point to note is that our parsers don't specify
which order -- Maybe/State or State/Maybe -- they prefer, and will 
work with either.  This means that the semantics are ultimately determined by 
the actual stack we use.




\section{Conclusion}
Monad transformers provide an elegant solution to error reporting in parser
combinators.  Better yet, they are extensible -- we could go back and add more
features to our parsers using additional transformers, perhaps extending our
parser to save partial results, so that if there's an error, it reports the
progress that it had made before the error.  We could do that using a Writer
transformer.  Or we check to make sure that all variables are in scope when
they're used -- using a Reader transformer.  We could even capture whitespace
and comment tokens, instead of just throwing them away, in case they contained
valuable information.  And the best part of it is, most of the parsers we already
wrote would continue to work fine without needing any modifications.  We'd just
have to set up our transformer stack and modify the relevant parsers! 




\bibliography{article}

\end{document}
